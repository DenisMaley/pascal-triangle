\documentclass[12pt]{article}
\usepackage{fontspec}
\usepackage{polyglossia}
\usepackage{empheq}
\usepackage{hyperref}
\usepackage{amssymb}
\usepackage{array}% http://ctan.org/pkg/array
\hypersetup{
  colorlinks=true,
  linkcolor=blue,
  urlcolor=blue
}
\setdefaultlanguage{english}
\setmainfont[Mapping=tex-text]{CMU Serif}

\title{Pascal's triangle divisibility problem}
\author{
Denis Maley  \\
Amsterdam  \\
}

\date{\today}

\begin{document}

\maketitle

\textbf{\large The problem description:}

We can easily verify that none of the entries in the first seven rows of 
Pascal's triangle are divisible by 7:

$$
\begin{tabular}{rccccccccccccc}
$m=0$:&    &    &    &    &    &    &  1\\\noalign{\smallskip\smallskip}
$m=1$:&    &    &    &    &    &  1 &    &  1\\\noalign{\smallskip\smallskip}
$m=2$:&    &    &    &    &  1 &    &  2 &    &  1\\\noalign{\smallskip\smallskip}
$m=3$:&    &    &    &  1 &    &  3 &    &  3 &    &  1\\\noalign{\smallskip\smallskip}
$m=4$:&    &    &  1 &    &  4 &    &  6 &    &  4 &    &  1\\\noalign{\smallskip\smallskip}
$m=5$:&    &  1 &    &  5 &    & 10 &    & 10 &    &  5 &    &  1\\\noalign{\smallskip\smallskip}
$m=6$:&  1 &    &  6 &    & 15 &    & 20 &    & 15 &    &  6 &    &  1\\\noalign{\smallskip\smallskip}
\end{tabular}
$$

However, if we check the first one hundred rows, we will find 
that only 2361 of the 5050 entries are \textit{not} divisible by 7.
Find the number of entries which are not divisible by 7 in the 
first one billion ($10^9$) rows of Pascal's triangle.

Let's reformulate the problem in general form:

\bigskip

\textit{
Given a non-negative integer $n$ and prime $p$, 
count the number of binomial coefficients $\binom m k$ 
for $m \le n$ that are not divisible by $p$.
}

\bigskip


\newpage

\textbf{\large Solution:}

\bigskip

Let's address to one a consequence of 
\href{https://en.wikipedia.org/wiki/Lucas%27s_theorem}{Lucas' theorem}:

\bigskip

A binomial coefficient $\binom m k$ is divisible by a prime $p$ 
($p$ divides $\binom m k$) if and only if at least one of the base $p$ digits 
of $k$ is greater than the corresponding digit of $m$.

\bigskip

Now let's agree on the overline notation used to express the base $p$ 
expansion:

$$n=n_ip^i+n_{i-1}p^{i-1}+\cdots +n_1p+n_0 = \overline{n_in_{i-1} \cdots n_0},$$
$$m=m_ip^i+m_{i-1}p^{i-1}+\cdots +m_1p+m_0 = \overline{m_im_{i-1} \cdots m_0},$$
$$k=k_ip^i+k_{i-1}p^{i-1}+\cdots +k_1p+k_0 = \overline{k_ik_{i-1} \cdots k_0}.$$

\bigskip

So, in other words, $p \mid \binom m k <=> \exists \: j: k_j > m_j$
\bigskip

Let's denote:

The amount of $\binom m k$ not divisible by $p$ in a row with
index $m$ as a function $d(m)$.

And the target function D(n) - the number of binomial coefficients $\binom m k$
for $m \le n$ that are not divisible by $p$:

$$D(n) = {\displaystyle \sum^{n}_{i=1}{d(i)}}$$

\bigskip

First we'll find d(m).
As an example, let's take a look on Pascal's triangle's row with index 6 and
consider $p = 3$: $m = 6 = 2*3 + 0 = \overline{20}$.


\begin{table}[!h]
  \centering
  \renewcommand{\arraystretch}{1.4}% Stretch tabular vertically
  \begin{tabular}{|c|m{1em}|m{1em}|m{1em}|m{1em}|m{1em}|m{1em}|m{1em}|}
    \hline
    decimal $k$ & 0 & 1 & 2 & 3 & 4 & 5 & 6 \\
    \hline
    $p$ expansion of $k$ & $\overline{00}$ & $\overline{01}$ & $\overline{02}$ &
    $\overline{10}$ & $\overline{11}$ & $\overline{12}$ &
    $\overline{20}$ \\
    \hline
    $j: k_j > m_j$ & $\nexists$ & 0 & 0 & $\nexists$ & 0 & 0 & $\nexists$ \\
    \hline
    $p \mid \binom m k$ & 0 & 1 & 1 & 0 & 1 & 1 & 0 \\
    \hline
    $\binom m k$ & 1 & 6 & 15 & 20 & 15 & 6 & 1 \\
    \hline
  \end{tabular}
\end{table}

So, in the row with index 6 we have $d(6) = 3$.
Now let's count this number without checking each coefficient in the row: for
$m = \overline{20}$, $\binom m k$ is not divisible if $k_1 \le 2$ and
$k_0 \le 0$
Consequently, $d(6) = (2 + 1) * (0 + 1)$

\bigskip

In general case for the $m = \overline{m_im_{i-1} \cdots m_0}$
$$d(m) = {\displaystyle \prod_{j=0}^{i} (m_j + 1)}$$

\bigskip

Now, we can notice the fractal structure of non-divisible by $p$ numbers in
the triangle, if we mark non-divisible as * and space for divisible ones.
Example for $p=3$:

$$
\begin{tabular}{rccccccccccccccccccc}
$m=0$:&&&&&&&&&&*\\\noalign{\smallskip\smallskip}
$m=1$:&&&&&&&&&*& &*\\\noalign{\smallskip\smallskip}
$m=2$:&&&&&&&&*& &*& &*\\\noalign{\smallskip\smallskip}
$m=3$:&&&&&&&*& & & & & &*\\\noalign{\smallskip\smallskip}
$m=4$:&&&&&&*& &*& & & &*& &*\\\noalign{\smallskip\smallskip}
$m=5$:&&&&&*& &*& &*& &*& &*& &*\\\noalign{\smallskip\smallskip}
$m=6$:&&&&*& & & & & &*& & & & & &*\\\noalign{\smallskip\smallskip}
$m=7$:&&&*& &*& & & &*& &*& & & &*& &*\\\noalign{\smallskip\smallskip}
$m=8$:&&*& &*& &*& &*& &*& &*& &*& &*& &*\\\noalign{\smallskip\smallskip}
\end{tabular}
$$

So for the first $p^i$ rows we'll have a triangle constructed by
${\displaystyle \frac{p(p+1)}{2}}$ (sum of arithmetic progression)
triangles of size $p^{i-1}$

\bigskip

Thus,

$$
D(p^i-1) = {\displaystyle \frac{p(p+1)}{2}} D(p^{i-1}-1) =
{\displaystyle \left( \dfrac{p(p+1)}{2} \right)}^i
$$

\newpage

Now we'll extend this result for a multiple of $p^i$:
$$n = n_ip^i - 1 = \overline{n_i0 \cdots 0} - 1$$

$$
D(n_ip^i-1) = {\displaystyle \frac{n_i(n_i+1)}{2}} D(p^{i}-1) =
{\displaystyle \frac{n_i(n_i+1)}{2} \left( \dfrac{p(p+1)}{2} \right)}^i
$$

In turn, we can now expand this formula to arbitrary $n$ if we notice that

$$n=n_ip^i + \overline{n_{i-1} \cdots n_0}$$

$$
D(\overline{n_in_{i-1} \cdots n_0}) =
D(\overline{n_i0 \cdots 0} - 1) + d(\overline{n_i0 \cdots 0}) + \cdots +
d(\overline{n_in_{i-1} \cdots n_0}) =
$$

$$
D(n_ip^i-1) + (n_i - 1)(d(\overline{0 \cdots 0}) + \cdots +
d(\overline{n_{i-1} \cdots n_0})) =
$$

$$
{\displaystyle \frac{n_i(n_i+1)}{2} \left( \dfrac{p(p+1)}{2} \right)}^i +
 (n_i - 1)D(\overline{n_{i-1} \cdots n_0})
$$

\bigskip

This formula is implemented in the 
\href{https://github.com/DenisMaley/pascal-triangle}{CLI app}.

\bigskip
\end{document}
